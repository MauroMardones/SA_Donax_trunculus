% Options for packages loaded elsewhere
\PassOptionsToPackage{unicode}{hyperref}
\PassOptionsToPackage{hyphens}{url}
\PassOptionsToPackage{dvipsnames,svgnames,x11names}{xcolor}
%
\documentclass[
]{article}
\usepackage{amsmath,amssymb}
\usepackage{iftex}
\ifPDFTeX
  \usepackage[T1]{fontenc}
  \usepackage[utf8]{inputenc}
  \usepackage{textcomp} % provide euro and other symbols
\else % if luatex or xetex
  \usepackage{unicode-math} % this also loads fontspec
  \defaultfontfeatures{Scale=MatchLowercase}
  \defaultfontfeatures[\rmfamily]{Ligatures=TeX,Scale=1}
\fi
\usepackage{lmodern}
\ifPDFTeX\else
  % xetex/luatex font selection
\fi
% Use upquote if available, for straight quotes in verbatim environments
\IfFileExists{upquote.sty}{\usepackage{upquote}}{}
\IfFileExists{microtype.sty}{% use microtype if available
  \usepackage[]{microtype}
  \UseMicrotypeSet[protrusion]{basicmath} % disable protrusion for tt fonts
}{}
\makeatletter
\@ifundefined{KOMAClassName}{% if non-KOMA class
  \IfFileExists{parskip.sty}{%
    \usepackage{parskip}
  }{% else
    \setlength{\parindent}{0pt}
    \setlength{\parskip}{6pt plus 2pt minus 1pt}}
}{% if KOMA class
  \KOMAoptions{parskip=half}}
\makeatother
\usepackage{xcolor}
\usepackage[margin=1in]{geometry}
\usepackage{color}
\usepackage{fancyvrb}
\newcommand{\VerbBar}{|}
\newcommand{\VERB}{\Verb[commandchars=\\\{\}]}
\DefineVerbatimEnvironment{Highlighting}{Verbatim}{commandchars=\\\{\}}
% Add ',fontsize=\small' for more characters per line
\usepackage{framed}
\definecolor{shadecolor}{RGB}{248,248,248}
\newenvironment{Shaded}{\begin{snugshade}}{\end{snugshade}}
\newcommand{\AlertTok}[1]{\textcolor[rgb]{0.94,0.16,0.16}{#1}}
\newcommand{\AnnotationTok}[1]{\textcolor[rgb]{0.56,0.35,0.01}{\textbf{\textit{#1}}}}
\newcommand{\AttributeTok}[1]{\textcolor[rgb]{0.13,0.29,0.53}{#1}}
\newcommand{\BaseNTok}[1]{\textcolor[rgb]{0.00,0.00,0.81}{#1}}
\newcommand{\BuiltInTok}[1]{#1}
\newcommand{\CharTok}[1]{\textcolor[rgb]{0.31,0.60,0.02}{#1}}
\newcommand{\CommentTok}[1]{\textcolor[rgb]{0.56,0.35,0.01}{\textit{#1}}}
\newcommand{\CommentVarTok}[1]{\textcolor[rgb]{0.56,0.35,0.01}{\textbf{\textit{#1}}}}
\newcommand{\ConstantTok}[1]{\textcolor[rgb]{0.56,0.35,0.01}{#1}}
\newcommand{\ControlFlowTok}[1]{\textcolor[rgb]{0.13,0.29,0.53}{\textbf{#1}}}
\newcommand{\DataTypeTok}[1]{\textcolor[rgb]{0.13,0.29,0.53}{#1}}
\newcommand{\DecValTok}[1]{\textcolor[rgb]{0.00,0.00,0.81}{#1}}
\newcommand{\DocumentationTok}[1]{\textcolor[rgb]{0.56,0.35,0.01}{\textbf{\textit{#1}}}}
\newcommand{\ErrorTok}[1]{\textcolor[rgb]{0.64,0.00,0.00}{\textbf{#1}}}
\newcommand{\ExtensionTok}[1]{#1}
\newcommand{\FloatTok}[1]{\textcolor[rgb]{0.00,0.00,0.81}{#1}}
\newcommand{\FunctionTok}[1]{\textcolor[rgb]{0.13,0.29,0.53}{\textbf{#1}}}
\newcommand{\ImportTok}[1]{#1}
\newcommand{\InformationTok}[1]{\textcolor[rgb]{0.56,0.35,0.01}{\textbf{\textit{#1}}}}
\newcommand{\KeywordTok}[1]{\textcolor[rgb]{0.13,0.29,0.53}{\textbf{#1}}}
\newcommand{\NormalTok}[1]{#1}
\newcommand{\OperatorTok}[1]{\textcolor[rgb]{0.81,0.36,0.00}{\textbf{#1}}}
\newcommand{\OtherTok}[1]{\textcolor[rgb]{0.56,0.35,0.01}{#1}}
\newcommand{\PreprocessorTok}[1]{\textcolor[rgb]{0.56,0.35,0.01}{\textit{#1}}}
\newcommand{\RegionMarkerTok}[1]{#1}
\newcommand{\SpecialCharTok}[1]{\textcolor[rgb]{0.81,0.36,0.00}{\textbf{#1}}}
\newcommand{\SpecialStringTok}[1]{\textcolor[rgb]{0.31,0.60,0.02}{#1}}
\newcommand{\StringTok}[1]{\textcolor[rgb]{0.31,0.60,0.02}{#1}}
\newcommand{\VariableTok}[1]{\textcolor[rgb]{0.00,0.00,0.00}{#1}}
\newcommand{\VerbatimStringTok}[1]{\textcolor[rgb]{0.31,0.60,0.02}{#1}}
\newcommand{\WarningTok}[1]{\textcolor[rgb]{0.56,0.35,0.01}{\textbf{\textit{#1}}}}
\usepackage{longtable,booktabs,array}
\usepackage{calc} % for calculating minipage widths
% Correct order of tables after \paragraph or \subparagraph
\usepackage{etoolbox}
\makeatletter
\patchcmd\longtable{\par}{\if@noskipsec\mbox{}\fi\par}{}{}
\makeatother
% Allow footnotes in longtable head/foot
\IfFileExists{footnotehyper.sty}{\usepackage{footnotehyper}}{\usepackage{footnote}}
\makesavenoteenv{longtable}
\usepackage{graphicx}
\makeatletter
\newsavebox\pandoc@box
\newcommand*\pandocbounded[1]{% scales image to fit in text height/width
  \sbox\pandoc@box{#1}%
  \Gscale@div\@tempa{\textheight}{\dimexpr\ht\pandoc@box+\dp\pandoc@box\relax}%
  \Gscale@div\@tempb{\linewidth}{\wd\pandoc@box}%
  \ifdim\@tempb\p@<\@tempa\p@\let\@tempa\@tempb\fi% select the smaller of both
  \ifdim\@tempa\p@<\p@\scalebox{\@tempa}{\usebox\pandoc@box}%
  \else\usebox{\pandoc@box}%
  \fi%
}
% Set default figure placement to htbp
\def\fps@figure{htbp}
\makeatother
\setlength{\emergencystretch}{3em} % prevent overfull lines
\providecommand{\tightlist}{%
  \setlength{\itemsep}{0pt}\setlength{\parskip}{0pt}}
\setcounter{secnumdepth}{-\maxdimen} % remove section numbering
\usepackage{fancyhdr}
\pagestyle{fancy}
\fancyhf{}
\lfoot[\thepage]{}
\rfoot[]{\thepage}
\fontsize{12}{22}
\selectfont
\usepackage{booktabs}
\usepackage{longtable}
\usepackage{array}
\usepackage{multirow}
\usepackage{wrapfig}
\usepackage{float}
\usepackage{colortbl}
\usepackage{pdflscape}
\usepackage{tabu}
\usepackage{threeparttable}
\usepackage{threeparttablex}
\usepackage[normalem]{ulem}
\usepackage{makecell}
\usepackage{xcolor}
\usepackage{bookmark}
\IfFileExists{xurl.sty}{\usepackage{xurl}}{} % add URL line breaks if available
\urlstyle{same}
\hypersetup{
  colorlinks=true,
  linkcolor={blue},
  filecolor={Maroon},
  citecolor={Blue},
  urlcolor={Blue},
  pdfcreator={LaTeX via pandoc}}

\title{\includegraphics[width=10cm,height=\textheight,keepaspectratio]{IEO-logo2.png}}
\author{}
\date{\vspace{-2.5em}}

\begin{document}
\maketitle


\pagenumbering{gobble}

%\begin{titlepage}
\begin{flushleft}
\Large{\textbf{Supplementary Material 2}}\\
\vspace*{2\baselineskip}
\LARGE{\textbf{Length composition simulation analysis in the wedge clam \textit{Donax trunculus} fishery using an integrated model}}\\
\vspace*{5\baselineskip}
\Large{Project FEMP 04}\\
\vspace*{1\baselineskip}
\Large{Instituto Español de Oceanografía, Cádiz }\\
\vspace*{4\baselineskip}
\end{flushleft}
\begin{flushright}
\large{\textit{Mauricio Mardones}}\\
\large{\textit{Marina Delgado}}\\
\vspace*{1\baselineskip}
\normalsize{\textbf{Fecha}}\\
Abril, 2024
\end{flushright}

% \end{titlepage}


\hypersetup{linkcolor = black}
\newpage
\pagenumbering{roman}
%\tableofcontents
%\addcontentsline{toc}{section}{\contentsname}

\newpage



\pagenumbering{arabic}
\hypersetup{linkcolor = blue}

{
\hypersetup{linkcolor=}
\setcounter{tocdepth}{3}
\tableofcontents
}
\pagebreak

\section{Context}\label{context}

This document presents a simulation approach to assess variability in length composition data within the context of stock assessment using the Stock Synthesis (SS3) framework. Historical commercial length composition data is used to generate alternative distributions by introducing stochastic noise. These simulated datasets are intended to support model fitting under different assumptions.

This analysis forms part of a broader evaluation of the potential impacts of varying selectivity patterns in the wedge clam (\emph{Donax trunculus}) fishery. Specifically, it aims to assess how different assumed or estimated mean sizes of selectivity influence key population variables, such as spawning biomass, recruitment, and exploitation rate. The study is focused on the artisanal wedge clam fishery in the Gulf of Cádiz, Spain.

\section{Data Reading and Preparation}\label{data-reading-and-preparation}

We begin by reading the input files for a reference model (scenario S1):

\begin{Shaded}
\begin{Highlighting}[]
\NormalTok{dir1 }\OtherTok{\textless{}{-}}\NormalTok{ here}\SpecialCharTok{::}\FunctionTok{here}\NormalTok{(}\StringTok{"s1"}\NormalTok{)}
\NormalTok{start1 }\OtherTok{\textless{}{-}} \FunctionTok{SS\_readstarter}\NormalTok{(}\AttributeTok{file =} \FunctionTok{file.path}\NormalTok{(dir1, }\StringTok{"starter.ss"}\NormalTok{),}
    \AttributeTok{verbose =} \ConstantTok{FALSE}\NormalTok{)}
\NormalTok{dat1 }\OtherTok{\textless{}{-}} \FunctionTok{SS\_readdat}\NormalTok{(}\AttributeTok{file =} \FunctionTok{file.path}\NormalTok{(dir1, start1}\SpecialCharTok{$}\NormalTok{datfile),}
    \AttributeTok{verbose =} \ConstantTok{FALSE}\NormalTok{)}
\FunctionTok{options}\NormalTok{(}\AttributeTok{OutDec =} \StringTok{"."}\NormalTok{)}
\end{Highlighting}
\end{Shaded}

\subsection{Extract Original Length Composition Data}\label{extract-original-length-composition-data}

\begin{Shaded}
\begin{Highlighting}[]
\NormalTok{datacom }\OtherTok{\textless{}{-}}\NormalTok{ dat1}\SpecialCharTok{$}\NormalTok{lencomp }\SpecialCharTok{\%\textgreater{}\%}
    \FunctionTok{filter}\NormalTok{(fleet }\SpecialCharTok{==} \StringTok{"1"}\NormalTok{)}

\NormalTok{datacom\_long }\OtherTok{\textless{}{-}}\NormalTok{ datacom }\SpecialCharTok{\%\textgreater{}\%}
    \FunctionTok{pivot\_longer}\NormalTok{(}\AttributeTok{cols =} \FunctionTok{starts\_with}\NormalTok{(}\StringTok{"l"}\NormalTok{), }\AttributeTok{names\_to =} \StringTok{"talla"}\NormalTok{,}
        \AttributeTok{values\_to =} \StringTok{"proporcion"}\NormalTok{) }\SpecialCharTok{\%\textgreater{}\%}
    \FunctionTok{mutate}\NormalTok{(}\AttributeTok{talla =} \FunctionTok{as.numeric}\NormalTok{(}\FunctionTok{sub}\NormalTok{(}\StringTok{"\^{}l"}\NormalTok{, }\StringTok{""}\NormalTok{, talla)))}

\NormalTok{estructura }\OtherTok{\textless{}{-}}\NormalTok{ datacom\_long }\SpecialCharTok{\%\textgreater{}\%}
\NormalTok{    dplyr}\SpecialCharTok{::}\FunctionTok{select}\NormalTok{(}\SpecialCharTok{{-}}\NormalTok{proporcion)}
\NormalTok{tallas }\OtherTok{\textless{}{-}} \FunctionTok{sort}\NormalTok{(}\FunctionTok{unique}\NormalTok{(datacom\_long}\SpecialCharTok{$}\NormalTok{talla))}
\NormalTok{anos }\OtherTok{\textless{}{-}} \FunctionTok{unique}\NormalTok{(datacom\_long}\SpecialCharTok{$}\NormalTok{year)}
\end{Highlighting}
\end{Shaded}

\section{Simulation Methodology}\label{simulation-methodology}

We define a function to simulate length composition data using a Gaussian distribution with added stochasticity:

\[
P_i = \frac{f_i \cdot \varepsilon_i}{\sum_j f_j \cdot \varepsilon_j}, \quad f_i = \text{dnorm}(x_i; \mu, \sigma)
\]

Where:
- \(P_i\) is the normalized proportion for length bin \(i\).
- \(f_i\) is the Gaussian density.
- \(\varepsilon_i\) is multiplicative noise, \(\sim \mathcal{N}(1, \sigma^2_\varepsilon)\).

\begin{Shaded}
\begin{Highlighting}[]
\NormalTok{simular\_proporcion\_ruido }\OtherTok{\textless{}{-}} \ControlFlowTok{function}\NormalTok{(mu, sigma, tallas,}
    \AttributeTok{ruido\_sd =} \FloatTok{0.1}\NormalTok{) \{}
\NormalTok{    densidades }\OtherTok{\textless{}{-}} \FunctionTok{dnorm}\NormalTok{(tallas, }\AttributeTok{mean =}\NormalTok{ mu, }\AttributeTok{sd =}\NormalTok{ sigma)}
\NormalTok{    ruido }\OtherTok{\textless{}{-}} \FunctionTok{rnorm}\NormalTok{(}\FunctionTok{length}\NormalTok{(tallas), }\AttributeTok{mean =} \DecValTok{1}\NormalTok{, }\AttributeTok{sd =}\NormalTok{ ruido\_sd)}
\NormalTok{    densidades\_ruido }\OtherTok{\textless{}{-}} \FunctionTok{pmax}\NormalTok{(densidades }\SpecialCharTok{*}\NormalTok{ ruido, }\DecValTok{0}\NormalTok{)}
\NormalTok{    proporciones }\OtherTok{\textless{}{-}}\NormalTok{ densidades\_ruido}\SpecialCharTok{/}\FunctionTok{sum}\NormalTok{(densidades\_ruido)}
    \FunctionTok{return}\NormalTok{(proporciones)}
\NormalTok{\}}
\end{Highlighting}
\end{Shaded}

\section{Generate Simulated Datasets}\label{generate-simulated-datasets}

We simulate data using several mean values (\(\mu = 2.3, 2.4, 2.5, 2.6\)) with a fixed standard deviation (\(\sigma = 0.25\)).

\begin{Shaded}
\begin{Highlighting}[]
\NormalTok{mus }\OtherTok{\textless{}{-}} \FunctionTok{c}\NormalTok{(}\FloatTok{2.3}\NormalTok{, }\FloatTok{2.4}\NormalTok{, }\FloatTok{2.5}\NormalTok{, }\FloatTok{2.6}\NormalTok{)}
\NormalTok{sigma }\OtherTok{\textless{}{-}} \FloatTok{0.25}

\NormalTok{simulados }\OtherTok{\textless{}{-}} \FunctionTok{map\_dfr}\NormalTok{(mus, }\ControlFlowTok{function}\NormalTok{(mu) \{}
    \FunctionTok{map\_dfr}\NormalTok{(anos, }\ControlFlowTok{function}\NormalTok{(ano) \{}
\NormalTok{        proporciones }\OtherTok{\textless{}{-}} \FunctionTok{simular\_proporcion\_ruido}\NormalTok{(mu,}
\NormalTok{            sigma, tallas)}
\NormalTok{        estructura }\SpecialCharTok{\%\textgreater{}\%}
            \FunctionTok{filter}\NormalTok{(year }\SpecialCharTok{==}\NormalTok{ ano) }\SpecialCharTok{\%\textgreater{}\%}
            \FunctionTok{mutate}\NormalTok{(}\AttributeTok{proporcion =} \FunctionTok{rep}\NormalTok{(proporciones, }\FunctionTok{nrow}\NormalTok{(.)}\SpecialCharTok{/}\FunctionTok{length}\NormalTok{(tallas)),}
                \AttributeTok{grupo =} \FunctionTok{paste0}\NormalTok{(}\StringTok{"media\_"}\NormalTok{, mu), }\AttributeTok{year =}\NormalTok{ ano)}
\NormalTok{    \})}
\NormalTok{\})}
\end{Highlighting}
\end{Shaded}

\section{Combine and Visualize}\label{combine-and-visualize}

Combine simulated and original data:

\begin{Shaded}
\begin{Highlighting}[]
\NormalTok{originales }\OtherTok{\textless{}{-}}\NormalTok{ datacom\_long }\SpecialCharTok{\%\textgreater{}\%}
    \FunctionTok{mutate}\NormalTok{(}\AttributeTok{grupo =} \StringTok{"original"}\NormalTok{)}
\NormalTok{todo }\OtherTok{\textless{}{-}} \FunctionTok{bind\_rows}\NormalTok{(originales, simulados)}
\end{Highlighting}
\end{Shaded}

\subsubsection{Compute Weighted Mean Lengths}\label{compute-weighted-mean-lengths}

\[
\bar{L} = \frac{\sum_i L_i P_i}{\sum_i P_i}
\]

\begin{Shaded}
\begin{Highlighting}[]
\NormalTok{medias }\OtherTok{\textless{}{-}}\NormalTok{ todo }\SpecialCharTok{\%\textgreater{}\%}
    \FunctionTok{group\_by}\NormalTok{(grupo, year) }\SpecialCharTok{\%\textgreater{}\%}
    \FunctionTok{summarise}\NormalTok{(}\AttributeTok{media\_talla =} \FunctionTok{sum}\NormalTok{(talla }\SpecialCharTok{*}\NormalTok{ proporcion)}\SpecialCharTok{/}\FunctionTok{sum}\NormalTok{(proporcion),}
        \AttributeTok{.groups =} \StringTok{"drop"}\NormalTok{)}
\end{Highlighting}
\end{Shaded}

\subsubsection{Plot Distributions}\label{plot-distributions}

\begin{Shaded}
\begin{Highlighting}[]
\FunctionTok{ggplot}\NormalTok{(todo, }\FunctionTok{aes}\NormalTok{(}\AttributeTok{x =}\NormalTok{ talla, }\AttributeTok{y =}\NormalTok{ proporcion)) }\SpecialCharTok{+} \FunctionTok{geom\_col}\NormalTok{(}\AttributeTok{position =} \StringTok{"identity"}\NormalTok{,}
    \AttributeTok{alpha =} \FloatTok{0.5}\NormalTok{, }\AttributeTok{color =} \StringTok{"black"}\NormalTok{) }\SpecialCharTok{+} \FunctionTok{facet\_grid}\NormalTok{(year }\SpecialCharTok{\textasciitilde{}}
\NormalTok{    grupo, }\AttributeTok{scales =} \StringTok{"free\_y"}\NormalTok{) }\SpecialCharTok{+} \FunctionTok{geom\_vline}\NormalTok{(}\AttributeTok{data =}\NormalTok{ medias,}
    \FunctionTok{aes}\NormalTok{(}\AttributeTok{xintercept =}\NormalTok{ media\_talla), }\AttributeTok{color =} \StringTok{"red"}\NormalTok{, }\AttributeTok{linetype =} \StringTok{"dashed"}\NormalTok{) }\SpecialCharTok{+}
    \FunctionTok{labs}\NormalTok{(}\AttributeTok{title =} \StringTok{"Distribuciones de talla: datos originales y simulados por año y grupo"}\NormalTok{,}
        \AttributeTok{x =} \StringTok{"Talla"}\NormalTok{, }\AttributeTok{y =} \StringTok{"Proporción"}\NormalTok{) }\SpecialCharTok{+} \FunctionTok{theme\_minimal}\NormalTok{() }\SpecialCharTok{+}
    \FunctionTok{theme}\NormalTok{(}\AttributeTok{legend.position =} \StringTok{"none"}\NormalTok{)}
\end{Highlighting}
\end{Shaded}

\pandocbounded{\includegraphics[keepaspectratio]{Suppl_Mat_2_files/figure-latex/unnamed-chunk-7-1.pdf}}

\section{Format for SS3 Input}\label{format-for-ss3-input}

Prepare data in wide format for Stock Synthesis:

\begin{Shaded}
\begin{Highlighting}[]
\NormalTok{cols\_to\_select }\OtherTok{\textless{}{-}} \FunctionTok{intersect}\NormalTok{(}\FunctionTok{c}\NormalTok{(}\StringTok{"year"}\NormalTok{, }\StringTok{"month"}\NormalTok{, }\StringTok{"fleet"}\NormalTok{,}
    \StringTok{"sex"}\NormalTok{, }\StringTok{"part"}\NormalTok{, }\StringTok{"Nsamp"}\NormalTok{, }\StringTok{"talla"}\NormalTok{, }\StringTok{"proporcion"}\NormalTok{,}
    \StringTok{"grupo"}\NormalTok{), }\FunctionTok{names}\NormalTok{(todo))}

\NormalTok{datacom\_wide }\OtherTok{\textless{}{-}}\NormalTok{ todo }\SpecialCharTok{\%\textgreater{}\%}
\NormalTok{    dplyr}\SpecialCharTok{::}\FunctionTok{select}\NormalTok{(}\FunctionTok{all\_of}\NormalTok{(cols\_to\_select)) }\SpecialCharTok{\%\textgreater{}\%}
    \FunctionTok{pivot\_wider}\NormalTok{(}\AttributeTok{id\_cols =} \FunctionTok{c}\NormalTok{(}\StringTok{"year"}\NormalTok{, }\StringTok{"month"}\NormalTok{, }\StringTok{"fleet"}\NormalTok{,}
        \StringTok{"sex"}\NormalTok{, }\StringTok{"part"}\NormalTok{, }\StringTok{"Nsamp"}\NormalTok{, }\StringTok{"grupo"}\NormalTok{), }\AttributeTok{names\_from =}\NormalTok{ talla,}
        \AttributeTok{names\_prefix =} \StringTok{"l"}\NormalTok{, }\AttributeTok{values\_from =}\NormalTok{ proporcion,}
        \AttributeTok{values\_fill =} \DecValTok{0}\NormalTok{) }\SpecialCharTok{\%\textgreater{}\%}
    \FunctionTok{arrange}\NormalTok{(year, month)}
\end{Highlighting}
\end{Shaded}

\section{Export to Excel}\label{export-to-excel}

Export each group into a separate worksheet with numeric formatting:

\begin{Shaded}
\begin{Highlighting}[]
\NormalTok{wb }\OtherTok{\textless{}{-}} \FunctionTok{createWorkbook}\NormalTok{()}

\ControlFlowTok{for}\NormalTok{ (gr }\ControlFlowTok{in} \FunctionTok{unique}\NormalTok{(datacom\_wide}\SpecialCharTok{$}\NormalTok{grupo)) \{}
\NormalTok{    df\_grupo }\OtherTok{\textless{}{-}}\NormalTok{ datacom\_wide }\SpecialCharTok{\%\textgreater{}\%}
        \FunctionTok{filter}\NormalTok{(grupo }\SpecialCharTok{==}\NormalTok{ gr)}

    \FunctionTok{addWorksheet}\NormalTok{(wb, }\AttributeTok{sheetName =} \FunctionTok{as.character}\NormalTok{(gr))}
    \FunctionTok{writeData}\NormalTok{(wb, }\AttributeTok{sheet =} \FunctionTok{as.character}\NormalTok{(gr), df\_grupo,}
        \AttributeTok{keepNA =} \ConstantTok{TRUE}\NormalTok{)}
\NormalTok{\}}

\CommentTok{\# Establecer formato numérico con punto decimal}
\NormalTok{number\_format }\OtherTok{\textless{}{-}} \FunctionTok{createStyle}\NormalTok{(}\AttributeTok{numFmt =} \StringTok{"0.00"}\NormalTok{)}

\CommentTok{\# Aplicar formato a todas las hojas y columnas}
\CommentTok{\# numéricas}
\ControlFlowTok{for}\NormalTok{ (sheet }\ControlFlowTok{in} \FunctionTok{names}\NormalTok{(wb)) \{}
\NormalTok{    df }\OtherTok{\textless{}{-}}\NormalTok{ datacom\_wide }\SpecialCharTok{\%\textgreater{}\%}
        \FunctionTok{filter}\NormalTok{(grupo }\SpecialCharTok{==}\NormalTok{ sheet)}
\NormalTok{    num\_cols }\OtherTok{\textless{}{-}} \FunctionTok{which}\NormalTok{(}\FunctionTok{sapply}\NormalTok{(df, is.numeric))}
    \ControlFlowTok{if}\NormalTok{ (}\FunctionTok{length}\NormalTok{(num\_cols) }\SpecialCharTok{\textgreater{}} \DecValTok{0}\NormalTok{) \{}
        \FunctionTok{addStyle}\NormalTok{(wb, }\AttributeTok{sheet =}\NormalTok{ sheet, }\AttributeTok{style =}\NormalTok{ number\_format,}
            \AttributeTok{rows =} \DecValTok{2}\SpecialCharTok{:}\NormalTok{(}\FunctionTok{nrow}\NormalTok{(df) }\SpecialCharTok{+} \DecValTok{1}\NormalTok{), }\AttributeTok{cols =}\NormalTok{ num\_cols,}
            \AttributeTok{gridExpand =} \ConstantTok{TRUE}\NormalTok{)}
\NormalTok{    \}}
\NormalTok{\}}

\FunctionTok{saveWorkbook}\NormalTok{(wb, }\AttributeTok{file =} \StringTok{"data\_sim\_length.xlsx"}\NormalTok{, }\AttributeTok{overwrite =} \ConstantTok{TRUE}\NormalTok{)}
\end{Highlighting}
\end{Shaded}

\section{Conclusion}\label{conclusion}

This simulation approach allows the incorporation of uncertainty in length composition data, which can be used to test sensitivity of stock assessment models to assumptions on size structure. It can also help to test model robustness to different assumptions of central tendency and variability in input data.

\end{document}
